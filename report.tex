\documentclass[11pt]{article}
\usepackage[margin=1.25in]{geometry}
\usepackage{graphicx,float}
\usepackage{amsmath,amsthm}
\usepackage{subcaption} % For side-by-side figures

% For labelings/struct descriptions
\usepackage{blindtext}
\usepackage{scrextend}
\addtokomafont{labelinglabel}{\sffamily}

\usepackage{listings} % For source code
\usepackage{algorithm,algorithmicx,algpseudocode} % For algorithms

%opening
\title{Force Directed Drawing}
\author{Vincent La}

\begin{document}

% \tableofcontents
\maketitle

\section{Introduction}
Force directed algorithms attempt to draw graphs by relating them to some physical analogy. For example, we may view vertices as steel balls and the edges between them as springs. One of the earlier force directed algorithms, Tutte's Barycenter Algorithm, attempts to place a graph's nodes along it's "center of mass."

\section{Tutte's Barycenter Method}
An early force directed drawing method was Tutte's Barycenter Method. In this method, the force on every vertex is given by 

\[ F(v) = \sum_{(u, v) \in E} (p_u - p_v) \]

Hence, we can ...

\[
\begin{aligned}
    \sum_{(u, v) \in E} (x_u - x_v) &= 0 \\
    \sum_{(u, v) \in E} (y_u - y_v) &= 0 \\
\end{aligned}
\]

Which we may rewrite as

\[
\begin{aligned}
    \deg{(v)}x_v - \sum_{u \in N_1(v)} x_u &= \sum_{w \in N_0(v)} x^*_w \\
    \deg{(v)}y_v - \sum_{u \in N_1(v)} y_u &= \sum_{w \in N_0(v)} y^*_w
\end{aligned}
\]

These equations are linear, and the resulting matrix is diagonally dominant (see Example 1.1). This is because the diagonal consists of vertex degrees, while the other entries $a_{ij}$ are either -1's (if $x_i$ and $x_j$ are neighbors) or 0's if they aren't.

\subsection{Example: Hypercube}
A simple example for which Tutte's method gives aesthetically pleasing results is the hypercube.

\begin{figure}[H]
    \includegraphics[width=\linewidth]{"report/prism_4".pdf}
\end{figure}

In the image below, the hypercube is placed in 500 x 500 pixel grid. The grid is governed by a simple Cartesian coordinate system, where the top left and bottom right corners have coordinates $(-250, 0)$ and $(250, 250)$ respectively. Four vertices are fixed and laid out into a circle of radius 250 centered at the origin. Hence, the bulk of the work performed algorithm is done in placing the center four free vertices. Labeling the free vertices as $x_1, x_2, x_3, x_4$, we may represent the task of laying out the free vertices with this matrix

\[
\begin{bmatrix}
    3 & -1 & 0 & -1 \\
    -1 & 3 & -1 & 0 \\
    0 & -1 & 3 & -1 \\
    -1 & 0 & -1 & 3 \\
\end{bmatrix}
\begin{bmatrix} x_1 \\ x_2 \\ x_3 \\ x_4 \end{bmatrix} =
\begin{bmatrix} 0 \\ 250 \\ 0 \\ -250 \end{bmatrix}
\]

The solution to this matrix is given by $x_1 = x_3 = 0, x_2 = \frac{250}{3}, x_4 = -\frac{250}{3}$.

\subsection{Resolution}
One the the main drawbacks of this algorithm is potentially poor resolution, i.e. the more edges and vertices we add to our graph, the harder it becomes to distinguish the different features of our graph. This is demonstrated best by the prism graph.

\paragraph{Prism Graph}
The prism graph denoted $\Pi_{n}$ is constructed by taking the vertices and edges of an $n$-prism. If we look below, the distance between vertices gets increasingly smaller as $n$ increases.

\begin{figure}[H]
    \begin{subfigure}{.3\textwidth}
        \includegraphics[width=\linewidth]{"report/prism_4".pdf}
    \end{subfigure}
    \begin{subfigure}{.3\textwidth}
        \includegraphics[width=\linewidth]{"report/prism_5".pdf}
    \end{subfigure}
    \begin{subfigure}{.3\textwidth}
        \includegraphics[width=\linewidth]{"report/prism_6".pdf}
    \end{subfigure}
    \begin{subfigure}{.3\textwidth}
        \includegraphics[width=\linewidth]{"report/prism_7".pdf}
    \end{subfigure}
    \begin{subfigure}{.3\textwidth}
        \includegraphics[width=\linewidth]{"report/prism_8".pdf}
    \end{subfigure}
    \begin{subfigure}{.3\textwidth}
        \includegraphics[width=\linewidth]{"report/prism_9".pdf}
    \end{subfigure}
\caption{$\Pi_4$ through $\Pi_{9}$ as drawn by Tutte's algorithm. Notice that $\Pi_4$ is isomorphic to the hypercube $Q_2$}
\end{figure}

\begin{figure}[H]
    \begin{subfigure}{.5\textwidth}
        \includegraphics[width=\linewidth]{"report/prism_5".pdf}
    \end{subfigure}
    \begin{subfigure}{.5\textwidth}
        \includegraphics[width=\linewidth]{"report/prism_10".pdf}
    \end{subfigure}
    \begin{subfigure}{.5\textwidth}
        \includegraphics[width=\linewidth]{"report/prism_20".pdf}
    \end{subfigure}
    \begin{subfigure}{.5\textwidth}
        \includegraphics[width=\linewidth]{"report/prism_40".pdf}
    \end{subfigure}
    \caption{$\Pi_5, \Pi_{10}, \Pi_{20}$ and $\Pi_{40}$ as drawn by Tutte's algorithm}
\end{figure}

\paragraph{Theorem} For any pair of adjacent "free" vertices in the prism graph drawn in the unit square, the distance between them is $O(\frac{1}{n})$.

\begin{proof}
    Let $P_n$ be the prism graph on $n$ vertices and show that the distance between any two vertices is $O(\frac{1}{n})$ by showing that $\text{dist(}{p_i, p_{i+1})} \leq a \cdot \frac{1}{n}$ for some constant a. Let $a \geq \frac{2\sqrt{2}}{n}$.
    
    \bigskip
    
    First, for any $p_i$ we have
    \[
        p_i = \left(
            \frac{x_{i-1} + x_{i+1} + f_{ix}}{n} , % x
            \frac{y_{i-1} + y_{i+1} + f_{iy}}{n} % y
        \right)
    \]
    
    and by extension
    \[
        p_{i + 1} = \left(
        \frac{x_i + x_{i+2} + f_{i + 1x}}{n} , % x
        \frac{y_i + y_{i+2} + f_{i + 1y}}{n} % y
        \right)
    \]
    
    Thus,
    \[
    \text{dist(}{p_i, p_{i + 1}}) = \sqrt{
        \left(\frac{x_i + x_{i+2} + f_{i+1} - x_{i-1} - x_{i+1} - f_i}{n}\right)^2 +
        \left(\frac{y_i + y_{i+2} + f_{i+1} - y_{i-1} - y_{i+1} - f_i}{n}\right)^2
    }
    \]
    
    So simplifying this gives,
    
    \[\begin{aligned}
    \text{dist(}{p_i, p_{i + 1}})
    &= \sqrt{
        \left(\frac{x_{i+2} + f_{i+1}}{n}\right)^2 +
        \left(\frac{y_{i+2} + f_{i+1}}{n}\right)^2
        } \\
    &= \sqrt{
        \left(\frac{2}{n}\right)^2 +
        \left(\frac{2}{n}\right)^2
    } 
    &\text{Because $P$ is drawn within unit square} \\
    &= \sqrt{\frac{4}{n^2} + \frac{4}{n^2}} \\
    &= \sqrt{\frac{8}{n^2}} \\
    &= 2\sqrt{2}\sqrt{\frac{1}{n^2}} \\
    &= \frac{2\sqrt{2}}{n} \leq a \cdot \frac{1}{n}
    \end{aligned}\]
    as desired.    
\end{proof}

\end{document}
